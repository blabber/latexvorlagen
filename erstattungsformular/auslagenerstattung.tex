\documentclass[ngerman,a4wide]{scrartcl}
\usepackage[utf8]{inputenc}
\usepackage[T1]{fontenc}
\usepackage{textcomp}
\usepackage{mathptmx}
\usepackage[scaled=.92]{helvet}
\usepackage{courier}
\renewcommand*{\familydefault}{phv}
\usepackage[left=25mm,top=25mm,bottom=10mm,right=10mm]{geometry}
\usepackage{fancyhdr}
\lhead{RaumZeitLabor e.V.}\chead{}\rhead{Formular C22 - Formblatt Auslagenerstattung\tiny{V2020.01.07}}
\lfoot{}\cfoot{}\rfoot{}
\pagestyle{fancy}
\usepackage{graphicx}
\usepackage{color}
\usepackage{floatflt}
\usepackage[
  pdftex,colorlinks=true,
  pdftitle={Formular C22},pdfsubject={Auslagenerstattung},
  pdfauthor={Laura Eckardt, Ranlvor},
  pdfpagemode=UseNone,pdfstartview=FitH,pagebackref,pdfhighlight={/N}, unicode=true
]{hyperref}
\newcommand{\textforlabel}[2]{%
\TextField[name={#1},value={#2},width=7em,align=2,%
bordercolor={1 1 1},readonly=true]{}%
}
\renewcommand{\baselinestretch}{1}
\begin{document}
\begin{floatingfigure}{0.45\textwidth}
    \vspace{3.5cm}
    \includegraphics[width=7cm]{RaumZeitLabor_Logo_Schwarz.pdf} %https://wiki.raumzeitlabor.de/wiki/Datei:RaumZeitLabor_-_Logo_-_Schwarz.svg
\end{floatingfigure}
\section*{Erstattung von Auslagen}
\begin{Form}
\hfill
\begin{tabular}{|rl|}
\hline
&\\*[-0.9em]\multicolumn{2}{|c|}{\textbf{Für den Kassenwart:}}\\
&\\*[-0.9em]Nummer:&%
\TextField[name=bnummer,width=15em,%
bordercolor={0.65 0.79 0.94},readonly=true]{}\\
&\\*[-0.9em]Datum:&%
\TextField[name=bdatum,width=15em,%
bordercolor={0.65 0.79 0.94},readonly=true]{}\\
&\\
\hline
\end{tabular}

\vspace{4cm}

\hfill
\begin{tabular}{|ll|}
\hline
Datum&Betrag (€) \\
\TextField[name=datum1,width=8em,bordercolor={0.65 0.79 0.94}]{}&\TextField[name=betrag1,width=5em,bordercolor={0.65 0.79 0.94}]{} \\
\TextField[name=datum2,width=8em,bordercolor={0.65 0.79 0.94}]{}&\TextField[name=betrag2,width=5em,bordercolor={0.65 0.79 0.94}]{} \\
\TextField[name=datum3,width=8em,bordercolor={0.65 0.79 0.94}]{}&\TextField[name=betrag3,width=5em,bordercolor={0.65 0.79 0.94}]{} \\
\TextField[name=datum4,width=8em,bordercolor={0.65 0.79 0.94}]{}&\TextField[name=betrag4,width=5em,bordercolor={0.65 0.79 0.94}]{} \\
\TextField[name=datum5,width=8em,bordercolor={0.65 0.79 0.94}]{}&\TextField[name=betrag5,width=5em,bordercolor={0.65 0.79 0.94}]{} \\
\TextField[name=datum6,width=8em,bordercolor={0.65 0.79 0.94}]{}&\TextField[name=betrag6,width=5em,bordercolor={0.65 0.79 0.94}]{} \\
\TextField[name=datum7,width=8em,bordercolor={0.65 0.79 0.94}]{}&\TextField[name=betrag7,width=5em,bordercolor={0.65 0.79 0.94}]{} \\
\TextField[name=datum8,width=8em,bordercolor={0.65 0.79 0.94}]{}&\TextField[name=betrag8,width=5em,bordercolor={0.65 0.79 0.94}]{} \\
\TextField[name=datum9,width=8em,bordercolor={0.65 0.79 0.94}]{}&\TextField[name=betrag9,width=5em,bordercolor={0.65 0.79 0.94}]{} \\
\TextField[name=datum10,width=8em,bordercolor={0.65 0.79 0.94}]{}&\TextField[name=betrag10,width=5em,bordercolor={0.65 0.79 0.94}]{} \\
&\\
\multicolumn{1}{|r}{\textbf{Gesamtbetrag:}} & \TextField[name=betraggesamt,width=5em,bordercolor={0.65 0.79 0.94}]{} \\
& \\
\hline
\end{tabular}
\vspace{0.5cm}

\hfill
\begin{tabular}{|rl|}
\hline
&\\*[-0.9em]\multicolumn{2}{|c|}{\textbf{Ausgelegt von:}}\\
&\\*[-0.9em]Vorname:&%
\TextField[name=vorname,width=15em,%
bordercolor={0.65 0.79 0.94}]{}\\
&\\*[-0.9em]Name:&%
\TextField[name=name,width=15em,%
bordercolor={0.65 0.79 0.94}]{}\\
&\\*[-0.9em]\multicolumn{2}{|c|}{\textbf{Auszahlung}}\\
&\\*[-0.9em]IBAN:&%
\TextField[name=konto,width=15em,%
bordercolor={0.65 0.79 0.94}]{}\\
&\\*[-0.9em]BIC:&%
\TextField[name=blz,width=15em,%
bordercolor={0.65 0.79 0.94}]{}\\
&\\*[-0.9em]Bank:&%
\TextField[name=bank,width=15em,%
bordercolor={0.65 0.79 0.94}]{}\\

&\\
\hline
\end{tabular}

\vspace{1cm}


\vspace{-19cm}
\begin{tabular}{|p{8cm}|}
\hline
Beleg hier einkleben oder an das Blatt heften\\
*[65em] \\
\hline
\end{tabular}



\vfill


%\vspace{2cm}



\end{Form}
\end{document}
